\documentclass[french]{article}
\usepackage[utf8]{inputenc}
\usepackage{physics}
\usepackage[T1]{fontenc}
\usepackage[a4paper, inner=1.5cm, outer=3cm, top=2cm,
bottom=3cm, bindingoffset=1cm]{geometry}
\usepackage{siunitx}
\usepackage{amssymb}
\usepackage{xcolor}
\usepackage{fancyhdr}
\fancyhf{}
\pagestyle{fancy}
\rhead{\itshape A.REDJIL}
\lhead{Section B/Section A }
\chead{Resume des Formules de Relativité Restreinte}
\rfoot{Page \thepage}
\setlength\parindent{0pt}
\begin{document}
	\begin{center}
		\begin{tabular}{| c | c |}
			\hline
			\textbf{Titre} & \textbf{Expression}\\
			\hline
			La vitesse de la lumière est invariante&$c = 3\times10^8 \si{\frac{\meter}{\second}}$\\
			\hline
			Un événement repéré dans R&$ \underline{r}=\mqty[c t& x&y&z]^t $ \\
			\hline
			La Matrice de Lorentz &$L = \mqty(\gamma & -\gamma\beta & 0 & 0 \\ -\gamma\beta& \gamma&0&0\\0&0&1&0\\0&0&0&1)$\\
			\hline
			L'Inverse de la Matrice de Lorentz&$L^{-1} = \mqty(\gamma & \gamma\beta & 0 & 0 \\ \gamma\beta& \gamma&0&0\\0&0&1&0\\0&0&0&1)$\\
			\hline
			L'expression de $\underline{r'}$ dans le repéré R&$\underline{r'}=L\underline{r}$\\
			\hline
			L'expression de $\underline{r}$ dans le repéré R' &$\underline{r}=L^{-1}\underline{r'}$\\
			\hline
			L’intervalle $\Delta s ^2$(invariant)&$\Delta S^2=c^2 \Delta t ^2-L^2, L^2=\Delta x ^2 +\Delta y ^2 + \Delta z ^2$\\
		\hline
		Contraction des longueurs&$L=\frac{L_0}{\gamma}$\\
		\hline
	        Dilatation du temps&$\Delta t = \gamma \Delta t' = \gamma \tau$  \\
		\hline
		Temps propre& $dt=\gamma_ud\tau$\\
		\hline
		Transformation des vitesses&$u_x = \frac{u_x'+V}{1 +\frac{\beta}{c}u_x'},
				 u_y = \frac{u_y'}{\gamma(1 +\frac{\beta}{c}u_x')}, 
				 u_z = \frac{u_z'}{\gamma(1 +\frac{\beta}{c}u_x')}$\\
		\hline
Transformation des accélérations&
	$a_x' = \frac{a_x}{\gamma^3\left(1 - \frac{\beta}{c}u_x\right)^3},
	a_y' = \frac{a_y + a_x \frac{\frac{\beta}{c}u_y}{1-\frac{\beta}{c}u_x}}{\gamma ^2\left(1-\frac{\beta}{c}u_x\right)^2}, 
		a_z' = \frac{a_z + a_x \frac{\frac{\beta}{c}u_z}{1-\frac{\beta}{c}u_x}}{\gamma ^2\left(1-\frac{\beta}{c}u_x\right)^2}$	
\\
\hline
Effet Doppler&$\omega'=\gamma\omega\left(1-\beta\cos(\theta)\right),
	\tan(\theta')= \frac{\sin(\theta)}{\gamma\left(\cos(\theta)-\beta\right)}$\\
\hline
Aberration&n = $\frac{dN}{d\Omega}= n_0\frac{1}{\gamma^2\left(1-\beta\cos(\theta)^2\right)}$\\
\hline
Quadrivecteur d’onde&$\underline{k}= \mqty(\frac{\omega}{c} &\vec{k})^t$\\
\hline
Quadrivecteur impulsion&$\underline{p}=m \underline{u}=\mqty(\gamma_u mc & \gamma_u m\vec{u})^t=\mqty(\frac{E}{c}&\vec{p})^t$\\
\hline
Quadrivecteur force&$\underline{F}=\frac{d\underline{p}}{d\tau}= m \underline{a}= \mqty(\gamma_umc\frac{d \gamma_u}{dt}&\gamma_u \frac{d\vec{p}}{dt})^t =\mqty(\frac{\gamma_u}{c}\vec{F}\vec{u}&\gamma_u\vec{F})^t$\\
\hline
Energies&$T = \left(\gamma_u-1\right)mc^2$\\
\hline
Relations force-acceleration&$\vec{F}=m\gamma_u\left(\vec{a}+\gamma_u^2\frac{\vec{u}\vec{a}}{c^2}\right), F_{||}=m\gamma_u^3a_T, F_{\perp}= m\gamma_ua_N$\\
\hline
Transformation des forces&$\underline{F'}= L\underline{F}$\\
\hline
Collisions relativistes&$p_{avant} = p_{apres}$(pour
un système isolé.)\\
\hline
Quadri-gradient&$\underline{\nabla}=\mqty(\pdv{ct}&-\vec{\nabla})^t, (\underline{\nabla'}=L\underline{\nabla})$\\
\hline
Quadri-vecteur ! densité de courant&$\underline{J}= \rho_0\mqty(\gamma_uc&\gamma_u\vec{u})^t$\\
\hline
Quadripotentiel&$\underline{A}=\mqty(\frac{\phi}{c}& A)^t$\\
\hline
La jauge de Lorenz&$\underline{\nabla}.\underline{A}=0$\\
\hline
Lagrangien relativiste (Particule Libre)&$L_{Libre}= -mc^2\sqrt{1-\beta_u^2}$\\
\hline
Lagrangien d’Interaction&$L = L_{libre}+L_{inter}=-mc^2\sqrt{1-\beta_u^2}-q\left(\phi-\vec{A}.\vec{u}\right)$\\
\hline
L’Hamiltonien Relativiste&$H = \sqrt{m^2c^4+\left(\vec{\pi}-q\vec{A}\right)^2c^2}+q\phi$
\\
\hline



		
			
			
			
		\end{tabular}
	\end{center}
\underline{remarque:} $\mqty(a&b&c&d)^t= \mqty(a\\b\\c\\d)$ 
\begin{thebibliography}{9}
	\bibitem{Cours}
	Cours de Monsieur R.Chami
	\bibitem{Cours}
	Cours de Madame L.Bouzar
\end{thebibliography}
\end{document}
